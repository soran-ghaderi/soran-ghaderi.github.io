%-------------------------
% Resume in Latex
% Author: Soran Ghaderi
% General ML Engineer/Researcher Resume
%------------------------

\documentclass[letterpaper,11pt]{article}

\usepackage[left=0.5in,right=0.5in,top=0.5in,bottom=0.5in]{geometry}
\usepackage{titlesec}
\usepackage{enumitem}
% \usepackage[hidelinks]{hyperref}
\usepackage[colorlinks=true,linkcolor=blue,urlcolor=blue,citecolor=blue]{hyperref}

\usepackage{fontawesome5}
\usepackage{helvet}
\renewcommand{\familydefault}{\sfdefault}

\pagestyle{empty}
\raggedbottom
\raggedright

% Section formatting
\titleformat{\section}{\large\bfseries\scshape\raggedright}{}{0em}{}[\titlerule]
\titlespacing{\section}{0pt}{8pt}{4pt}

\begin{document}

%----------HEADER----------
\begin{center}
    \textbf{\Large Soran Ghaderi} \\
    \small \faIcon{envelope} \href{mailto:soran.gdr.cs@gmail.com}{soran.gdr.cs@gmail.com} $|$ 
    \faIcon{github} \href{https://github.com/soran-ghaderi}{GitHub} $|$ 
    \faIcon{linkedin} \href{https://www.linkedin.com/in/soran-ghaderi}{LinkedIn} $|$
    \faIcon{twitter} \href{https://twitter.com/soranghadri}{X} $|$ 
    \faIcon{globe} \href{https://soran-ghaderi.github.io/}{Website} $|$ 
    \faIcon{graduation-cap} \href{https://scholar.google.com/citations?user=-2N2iKcAAAAJ}{Scholar}
\end{center}

%-----------SUMMARY-----------
\section{Research Profile}


MSc Artificial Intelligence graduate (\textbf{Distinction, 4.0/4.0 GPA equivalent}) and research intern at UIUC Blender Lab. My research focuses on understanding when generative models should learn globally versus search locally at inference time---treating learning, sampling, and search as points on a continuous spectrum of measure transport operations. Currently preparing work for \textbf{ICML 2026} (with UIUC/Harvard) on mode collapse in energy-based transformers for high-dimensional multi-modal spaces.
\textbf{Relevant Experience:}
\begin{itemize}[leftmargin=15pt,topsep=2pt]
    \item Creator of \textbf{TorchEBM (8K+ downloads)}: PyTorch library for EBMs, diffusion, flow matching, SDE/ODE integrators
    \item Developer of \textbf{cuRBLAS}: CUDA library for randomized BLAS and probabilistic linear algebra operations
    \item Developing mathematical frameworks for training EBMs with long sampling trajectories via implicit differentiation
\end{itemize}

\textbf{Research Interests:} Unified theory of learning, sampling, and search through information geometry and optimal transport. Geometry-aware adaptive algorithms for measure transport. Applications to molecular sampling, world models, and compositional generation. \textbf{UK Home Fee Status}.



%-----------EDUCATION-----------
\section{Education}

\subsection*{\textbf{University of Essex} \hfill \textbf{Oct 2023 - Oct 2024}}
\textit{MSc Artificial Intelligence (\textbf{Distinction} - \textbf{4.0/4.0} GPA equivalent)} \hfill \textit{UK}
\begin{itemize}[leftmargin=15pt]
    \item Dissertation: \textbf{Neural Integration of Iterative Reasoning (\href{https://soran-ghaderi.github.io/nir/}{NIR}) in LLMs for Code Generation} (\textbf{90/100})
\end{itemize}

\subsection*{\textbf{University of Kurdistan} \hfill \textbf{2014 - 2018}}
\textit{BEng Computer Engineering - Software} \hfill \textit{Iran}
\begin{itemize}[leftmargin=15pt]
    \item Thesis: EfficientCoF - Subspace and K-Means Clustering-based Method for Collaborative Filtering (\textbf{19.5/20})
\end{itemize}

%-----------PUBLICATIONS-----------
\section{Publications \& Works in Progress}

\subsection*{\textbf{Energy Outscales Diffusion and Flow---Was More Supervision All We Needed?} \hfill \textbf{Dec 2025 and 2026}}
\textit{Alexi Gladstone, Shivanshu Shekhar, Avery Qian, Ninad Daithankar, \textbf{Soran Ghaderi}, Yilun Du, Heng Ji, Tong Zhang}
\\ \textit{Preprint under active writing, available soon. \textbf{In preparation for ICML 2026 submission}}
\begin{itemize}[leftmargin=15pt]
    \item Developed complete experimental \textbf{codebase} for \textbf{class-conditional image generation} with Energy-Based Transformers
    \item Implemented and evaluated multiple \textbf{EBT variants}: latent-variable conditioning, best-of-k sampling, depth-hierarchical architectures
    \item Engineered \textbf{latent caching system} and trained/monitored all models alongside diffusion transformer baselines
    \item Contributed to the theoretical framework \textbf{investigating mode mixing} in high-dimensional multi-modal spaces
\end{itemize}
\subsection*{\textbf{Implicit Gradients for Energy-Based Transformers} \hfill \textbf{Expected 2026}}
\begin{itemize}[leftmargin=15pt]
    \item \textbf{Ghaderi, S.} et al. \textit{collaboration with Oxford / UIUC / Harvard.}
\\ \textit{Work in Progress.}
\end{itemize}

\subsection*{\textbf{Hierarchical Energy Transformers}\hfill \textbf{Expected 2026}}
\begin{itemize}[leftmargin=15pt]
    \item \textbf{Ghaderi, S.} et al. \textit{collaboration with UIUC / Harvard.}
\\ \textit{Work in Progress.}
\end{itemize}



%-----------EXPERIENCE-----------
\section{Research and Development Experience}

\subsection*{\textbf{Research Intern - Energy-based Models (EBMs)} \hfill \textbf{Jul 2025 - Present}}
\textit{Blender Lab, University of Illinois Urbana-Champaign (UIUC)} \hfill \textit{ US-Remote}
\begin{itemize}[leftmargin=15pt]
    \item Scaling energy-based transformers for text-to-image and text-to-video generation
    \item Investigating inference-time compute scaling and fast sampling strategies for large-scale generative models
    \item Developing modality-agnostic reasoning frameworks applicable to world models and temporal prediction
\end{itemize}


\subsection*{\textbf{Developer - cuRBLAS CUDA Randomized BLAS Library} \hfill \textbf{2025 - Present}}
\textit{GitHub (\href{https://github.com/soran-ghaderi/cuRBLAS}{Repository})} \hfill \textit{C++, CUDA, Python}
\begin{itemize}[leftmargin=15pt]
    \item Developing modular GPU-accelerated randomized BLAS library with Python bindings optimized for large-scale AI workloads, and advanced CUDA kernels for probabilistic linear algebra operations
\end{itemize}


\subsection*{\textbf{Developer - TorchEBM (EBM, Diffusion) Library} \hfill \textbf{2024 - Present}}
\textit{GitHub (\href{https://github.com/soran-ghaderi/TorchEBM}{Repository} $|$ \href{https://soran-ghaderi.github.io/torchebm/}{Website}) - \textbf{7.3K+ downloads}} \hfill \textit{Python, PyTorch, CUDA}
\begin{itemize}[leftmargin=15pt]
    \item Built PyTorch library for GPU-accelerated energy-based models, diffusion processes, and flow matching, focusing on continuous generative modeling and real-time inference optimization
    \item Implemented efficient ODE/SDE solvers/integrators, samplers, losses, models, schedulers, and optimization algorithms through mixed precision training for fast generation, emphasizing memory efficiency and hardware-aware algorithm design. Equilibrium matching, new samplers, models, and transport plans to be featured in the next release.
\end{itemize}

\subsection*{\textbf{Postgraduate Researcher - LLM Code Generation} \hfill \textbf{Apr 2024 - Oct 2024}}
\textit{University of Essex, supervised by Prof. Luca Citi $|$ Website (\href{https://soran-ghaderi.github.io/nir/}{NIR})} \hfill \textit{UK}
\begin{itemize}[leftmargin=15pt]
    \item Developed Neural Integration of Iterative Reasoning (NIR) framework for code generation. Integrated reasoning contexts directly into hidden states without requiring model fine-tuning, designed separate deep-think stage with self-reflection mechanism, evaluated the results
    \item Implemented reinforcement learning agents for playing games in complex environments
    \item Built GPU-accelerated multimodal navigation system on Jetson hardware, optimizing inference for real-time concurrent visual and textual processing, addressing challenges in real-time multimodal AI deployment
\end{itemize}


\subsection*{\textbf{Lead Developer - High-Performance \& GPU Optimized ML Libraries} \hfill \textbf{2019 - 2024}}
\textit{GitHub (\href{https://github.com/tensorops/TransformerX}{TransformerX}, \href{https://github.com/bi-graph/Emgraph}{Emgraph}, \href{https://github.com/bi-graph/Bigraph}{Bigraph}, \href{https://github.com/soran-ghaderi/backpropagation}{Nano Autodiff})} \hfill \textit{Python, TensorFlow, PyTorch}
\begin{itemize}[leftmargin=15pt]
    \item Built and maintained open-source libraries: \textbf{TransformerX} (transformer research), \textbf{Emgraph} (knowledge-graph embeddings), \textbf{Bigraph} (bipartite link prediction), and \textbf{Nano Autodiff} (from-scratch autodiff with numerically stable computation graph).
    \item Established APIs, comprehensive documentation, and testing frameworks adopted by ML community


\end{itemize}

\subsection*{\textbf{Lead Developer - TASE Multimodal Search Engine} \hfill \textbf{2020 - 2022}}
\textit{GitHub (\href{https://github.com/appheap/TASE}{Repository})} \hfill \textit{Python, Elasticsearch}
\begin{itemize}[leftmargin=15pt]
    \item Architected large-scale feature-rich audio search engine with efficient indexing and retrieval pipeline with scalable backend using Elasticsearch, ArangoDB, and Redis for real-time audio and text processing
\end{itemize}

% ---------- Technical Writing ------------
\subsection*{\textbf{Technical Writing} \hfill \textbf{Oct 2022 - Present}}
\textit{Blog (\href{https://soran-ghaderi.github.io/blog}{soran-ghaderi.github.io/blog})} \hfill \textit{Medium, Towards Data Science, TorchEBM Blog}
\begin{itemize}[leftmargin=15pt]
    \item Published technical articles on Hamiltonian mechanics, physics, samplers, transformers, attention mechanisms, generative models, and API documentation of multiple ML libraries, reaching a wide ML practitioner audience (\textbf{50K+ views on Medium alone})
\end{itemize}

%-----------AWARDS-----------
\section{Awards \& Recognition}
\subsection*{\textbf{Postgraduate Scholarship} \hfill \textbf{Oct 2023 - Oct 2024}}
\textit{Full Scholarship - University of Essex} \hfill \textit{UK}
\begin{itemize}[leftmargin=15pt]
    \item Awarded a \textbf{full scholarship} for a record of academic excellence and potential for AI research success
\end{itemize}
\subsection*{\textbf{Neural Network Design Challenge - Rossmann (Kaggle)} \hfill \textbf{Oct 2023}}
\begin{itemize}[leftmargin=15pt]
    \item \textbf{First Place} (100+ participants, inter-departmental) - scores: 0.10886 (public) / 0.11384 (private) \hfill \textit{UK}
\end{itemize}



%-----------SKILLS-----------
\section{Technical Skills}

% ===============================================

\textbf{Deep Learning \& Generative Models:} Energy-Based Models, Diffusion Models, Flow Matching, LLMs, Transformer Architectures, Vision-Language Models, Image/Video Generation, Reinforcement Learning, Multimodal Learning, MCMC Sampling

\textbf{Programming \& ML Frameworks:} Python, PyTorch (advanced), CUDA C/C++, JAX, TensorFlow, NumPy, Triton, Custom Kernel Development, Performance Profiling

\textbf{GPU Computing \& HPC:} Multi-GPU Training, Distributed Computing, Mixed Precision (FP16/BF16), Quantization, Kernel Optimization, Pipeline Parallelization, Slurm, Model Deployment

\textbf{Mathematics \& Theory:} Linear Algebra, Optimisation Theory, Calculus, Probability, Statistics, Numerical Computing, Differential Equations, MCMC, Familiar with Information Theory \& Statistical Mechanics \\

\textbf{Multimodal \& Cross-Modal Learning:} Domain Adaptation, Few-shot Learning, Attention Mechanisms, Multi-task Learning, Inference-time Compute Scaling, RLHF, SFT \\

\textbf{Software Engineering \& System Design:} ML Pipelines, Scalable Architecture Design, Version Control (Git), CI/CD, Testing Frameworks, Containerisation, Performance Measurement, Code Reviews, Agile Methodologies \\

%-----------RESEARCH INTERESTS-----------
\section{Research Interests}

\textbf{Learning-Sampling-Search Tradeoff:} When should generative models search at inference versus directly sample? Deriving information-geometric criteria (score divergence, Fisher information, transport cost) that separate regimes where learned transport suffices from those requiring active exploration.
\\
\textbf{Geometry-Aware Adaptive Transport:} Extending SMC and Feynman-Kac correctors to learned Riemannian manifolds. Developing algorithms that allocate computation dynamically between global transport and local refinement based on target geometry.
\\
\textbf{AI4Science Applications:} Molecular conformational sampling, equilibrium sampling via Sequential Boltzmann Generators, free energy calculations. Structure-preserving models respecting geometric constraints.
\\
\textbf{Generalisation via Inference-Time Search:} Meta-transport maps that generalise across target families without retraining. Compositional generation combining transport primitives for unseen targets through structured search.
\\


%-----------REFERENCES-----------
\section{References}

\subsection*{\textbf{Professor Luca Citi}}
\textit{Professor, School of Computer Science and Electronic Engineering (CSEE)} \\
\textit{University of Essex, Colchester Campus, 1NW.5.3C} \\
Email: lciti@essex.ac.uk $|$ Tel: +44 (0)1206 874233 \\
\textit{Dissertation Supervisor (MSc Artificial Intelligence)}

\subsection*{\textbf{Dr Michael Fairbank}}
\textit{Senior Lecturer, School of Computer Science and Electronic Engineering (CSEE)} \\
\textit{University of Essex, Colchester Campus, 1NW.3.19} \\
Email: m.fairbank@essex.ac.uk  \\
\textit{Game AI Module Lecturer (MSc Artificial Intelligence)}

\subsection*{\textbf{Dr Şefki Kolozali}}
\textit{Lecturer, School of Computer Science and Electronic Engineering (CSEE)} \\
\textit{University of Essex, Colchester Campus, 5A.523} \\
Email: sefki.kolozali@essex.ac.uk $|$ Tel: +44 (0) 1206 873302 \\
\textit{Group Project Supervisor (MSc Artificial Intelligence)}

\end{document}
